
\section{Study}

First we define a 'Functional Regex'(FR) as some regex that performs in a specific way.  For many FRs, there are several concrete ways to express a single FR.
We define a concrete regex(CR) as a regex expressed with a particular pattern String.
Here is one illustration of these definitions:

\todoNow{create some examples for these terms}

We identified 10 loose groups of FRs, described in this table:

\todoNow{create a table explaining the 10 groups}

For each of these groups we created either two concrete versions of three FRs or three concrete versions of two FRs.

Each of the 10 categories had 6 concrete versions of some FR and so there are 60 CRs.  For each CR, we selected 5 \emph{example strings} designed to test the understanding of the CR.  The idea is that different CRs may have different levels of readability, even when they are representing the same FR.  We define readability as the ability to look at the CR and determine if an \emph{example string} can be matched by it or not.

\todoNow{create some illustration of one matching subtask}

In mechanical Turk, we designed a 180 tasks composed of 10 matching subtasks, so that each of the 60 CRs had 30 separate observations (each an average of 5 \emph{example string} problems).  These 1800 observations are what the analysis will focus on.



